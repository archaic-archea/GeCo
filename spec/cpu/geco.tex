\documentclass[6pt]{article}

\def\version{0.1 Rev 2.0}
\def\name{Germanium Core}
\def\sname{GeCo}

\usepackage{tabularx}
\usepackage{bytefield}

\title{\name}
\author{Lilly Anderson}

\begin{document}
\maketitle

\pagebreak
\tableofcontents

\pagebreak
\addcontentsline{toc}
{section}{Introduction}
\section*{Introduction}
This is the \sname\ (\name) architectural specification v\version. This gives information on the hardware expectations of a valid \sname\ implementation.

\subsection{Goals}
\begin{itemize}
    \item Symmetric Multi-Processing
    \item Variable instruction length (16, 32, and 64 bit sizes)
    \item Software managed address translation
    \item Memory mapped IO
\end{itemize}

\section{Hints}
For GeCo processors there are provided hints to help communicate to the prcoessor things about performance related issues. These hints aren't required to have any effect on architectural state\footnote{There are important exceptions to this, these cases will be explicitly mentioned when they come up.}, but must not generate a fault if their associated extension is implemented.

\section{Encodings}
Different encodings can be used in GeCo to reduce code size, and increase code efficiency. Every given version of GeCo can have its own required encodings, as well as a list of optional encodings, this helps keep standards for programms so they can quickly know if their program is supported.

\section{16 Bit Encodings}
This section describes compressed instruction encodings for \sname\ to help reduce binary sizes. We will use "CGC" to refer to 16 bit encodings in general.

\subsection{Overview}
CGC is used to compress a few different kinds of instructions, specifically common types of instructions, such as stack, thread, and global relative accesses, loading immediates, arithmetic operations, jumping, and branching. It supports 5 different encodings.
\begin{itemize}
    \item RelIAcc Encoding (encoding 0b01), handles relative accesses for the SP, TP, and GP registers with an offset from an immediate
    \item ARAcc Encoding (encoding 0b00), handles Arithmetic accesses between 2 registers
    \item AIAcc Encoding (encoding 0b10), handles Arithmetic accesses with 1 register and an immediate
    \item Ctrl Encoding (encoding 0b11), instructions for controlling or hinting things to the processor
    \item 
\end{itemize}
\subsection{RelIAcc}
The RelIAcc encoding handles relative memory accesses using an immediate as offset.

\begin{center}
\begin{tabularx}{\textwidth}{ |>{\raggedright\arraybackslash}X|>{\raggedright\arraybackslash}X|>{\raggedright\arraybackslash}X|>{\raggedright\arraybackslash}X| }
    \hline
    Instruction mneumonic & Use & Format & Opcode \\
    \hline
    cgc.ld & Loads 2 bytes & cgc.ld rd, imm(ra) & 0b00 \\
    \hline
    cgc.lo & Loads 8 bytes & cgc.lo rd, imm(ra) & 0b01 \\
    \hline
    cgc.sd & Stores 2 bytes & cgc.sd rs, imm(ra) & 0b10 \\
    \hline
    cgc.so & Stores 8 bytes & cgc.so rs, imm(ra) & 0b111 \\
    \hline
\end{tabularx}
\end{center}

\begin{center}
\begin{tabularx}{\textwidth}{ |>{\raggedright\arraybackslash}X|>{\raggedright\arraybackslash}X|>{\raggedright\arraybackslash}X| }
    \hline
    Bit range (inclusive) & Name & Abbreviation \\
    \hline
    0..1 & Primary Encoding Size & n/a \\
    \hline
    2..3 & Encoding & n/a \\
    \hline
    4..5 & Opcode & n/a \\
    \hline
    6..7 & Address IDR & ra \\
    \hline
    8..10 & Source/Destination HGP Register & rs/rd \\
    \hline
    11..15 & Immediate Offset Bits 1..5 & imm \\
    \hline
\end{tabularx}
\end{center}

\subsection{ARAcc}
The ARAcc encoding handles arithmetic operations between 2 registers.

\begin{center}
\begin{tabularx}{\textwidth}{ |>{\raggedright\arraybackslash}X|>{\raggedright\arraybackslash}X|>{\raggedright\arraybackslash}X|>{\raggedright\arraybackslash}X| }
    \hline
    Instruction mneumonic & Use & Format & Opcode \\
    \hline
    cgc.add & Adds r\textsubscript{0} into r\textsubscript{1} & cgc.add r\textsubscript{0}, r\textsubscript{1} & 0b0001 \\
    \hline
    cgc.sub & Subtracts r\textsubscript{0} from r\textsubscript{1} & cgc.sub r\textsubscript{0}, r\textsubscript{1} & 0b0010 \\
    \hline
    cgc.mul & Multiplies r\textsubscript{1} by r\textsubscript{0} & cgc.mul r\textsubscript{0}, r\textsubscript{1} & 0b0011 \\
    \hline
    cgc.div & Divides r\textsubscript{1} by r\textsubscript{0} & cgc.div r\textsubscript{0}, r\textsubscript{1} & 0b0100 \\
    \hline
    cgc.not & Inverts r\textsubscript{0} and r\textsubscript{1} & cgc.not r\textsubscript{0}, r\textsubscript{1} & 0b0101 \\
    \hline
    cgc.srl & Shifts r\textsubscript{1} left by r\textsubscript{0} bits & cgc.srl r\textsubscript{0}, r\textsubscript{1} & 0b0110 \\
    \hline
    cgc.srr & Shifts r\textsubscript{1} right by r\textsubscript{0} bits & cgc.srr r\textsubscript{0}, r\textsubscript{1} & 0b0111 \\
    \hline
    cgc.xor & Xor's r\textsubscript{1} with r\textsubscript{0} & cgc.xor r\textsubscript{0}, r\textsubscript{1} & 0b1000 \\
    \hline
    cgc.and & And's r\textsubscript{1} with r\textsubscript{0} & cgc.and r\textsubscript{0}, r\textsubscript{1} & 0b1001 \\
    \hline
    cgc.or & Or's r\textsubscript{1} with r\textsubscript{0} & cgc.or r\textsubscript{0}, r\textsubscript{1} & 0b1010 \\
    \hline
\end{tabularx}
\end{center}

\begin{center}
\begin{tabularx}{\textwidth}{ |>{\raggedright\arraybackslash}X|>{\raggedright\arraybackslash}X|>{\raggedright\arraybackslash}X| }
    \hline
    Bit range (inclusive) & Name & Abbreviation \\
    \hline
    0..1 & Primary Encoding Size & n/a \\
    \hline
    2..3 & Encoding & n/a \\
    \hline
    4..7 & Opcode & n/a \\
    \hline
    8..9 & Reserved & n/a \\
    \hline
    10..12 & HGP Reg\textsubscript{0} & r\textsubscript{0} \\
    \hline
    13..15 & HGP Reg\textsubscript{1}/Destination Register & r\textsubscript{1}/rd \\
    \hline
\end{tabularx}
\end{center}

\subsection{AIAcc}
The AIAcc encoding handles arithmetic operations between 1 register, and an immediate.

\begin{center}
\begin{tabularx}{\textwidth}{ |>{\raggedright\arraybackslash}X|>{\raggedright\arraybackslash}X|>{\raggedright\arraybackslash}X|>{\raggedright\arraybackslash}X| }
    \hline
    Instruction mneumonic & Use & Format & Opcode \\
    \hline
    cgc.addiu & Adds a non-sign extended imm to rd & cgc.addiu rd, imm & 0b000 \\
    \hline
    cgc.addi & Adds a sign extended imm to rd & cgc.addi rd, imm & 0b001 \\
    \hline
    cgc.mul & Multiplies rd by a sign extended imm & cgc.mul rd, imm & 0b010 \\
    \hline
    cgc.div & Divides rd by a sign extended imm & cgc.div rd, imm & 0b011 \\
    \hline
    cgc.srl & Shifts rd left by imm bits & cgc.srl rd, imm & 0b100 \\
    \hline
    cgc.srr & Shifts rd right by imm bits & cgc.srr rd, imm & 0b101 \\
    \hline
    cgc.xor & Xor's rd with a sign extended imm & cgc.xor rd, imm & 0b110 \\
    \hline
    cgc.or & Or's rd with a sign extended imm & cgc.or rd, imm & 0b111 \\
    \hline
\end{tabularx}
\end{center}

\begin{center}
\begin{tabularx}{\textwidth}{ |>{\raggedright\arraybackslash}X|>{\raggedright\arraybackslash}X|>{\raggedright\arraybackslash}X| }
    \hline
    Bit range (inclusive) & Name & Abbreviation \\
    \hline
    0..1 & Primary Encoding Size & n/a \\
    \hline
    2..3 & Encoding & n/a \\
    \hline
    4..6 & Opcode & n/a \\
    \hline
    7..9 & HGP Destination Register & rd \\
    \hline
    10..15 & Immediate Bits 0..5 & imm \\
    \hline
\end{tabularx}
\end{center}

\subsection{Ctrl}
The cgc extension provides smaller ways of using hints, and control instructions to reduce code footprint. The existence of processor hints are explained in section 1.

\begin{center}
\begin{tabularx}{\textwidth}{ |>{\raggedright\arraybackslash}X|>{\raggedright\arraybackslash}X| }
\end{tabularx}
\end{center}

\begin{center}
\begin{tabularx}{\textwidth}{ |>{\raggedright\arraybackslash}X|>{\raggedright\arraybackslash}X|>{\raggedright\arraybackslash}X| }
    \hline
    Bit range (inclusive) & Name & Abbreviation \\
    \hline
    0..1 & Primary Encoding Size & n/a \\
    \hline
    2..3 & Encoding & n/a \\
    \hline
    4..15 & Opcode & n/a \\
    \hline
\end{tabularx}
\end{center}

\section{32 Bit Encodings}
The standard instruction size that must be supported by all implementations is 32 bits, this is to provide enough space for needed encodings, while avoiding taking up too much memory.
Being larger than CGC, the standard 32 bit encoding is a lot more complex, and features way more twisty aspects of it's design to help keep things nice.

\subsection{Encodings}
Provided here is a list of different encodings, each will have their own diagram to attempt to show off how it's formatted.

\subsection{RTRA}
\resizebox{\textwidth}{!}{%
    \begin{bytefield}[boxformatting={\centering\small}, bitwidth=auto, endianness=little]{64}
        \bitheader{63, 0} \\
        \bitbox{64}{Main field}
    \end{bytefield}
}

\section{64 Bit Encodings}

\section{Past 64 Bits}
Although encodings above 64 bits are not supported at this time, we would like to provide a framework for future designs for larger encoding sizes in case the need arises. The primary encoding size field should always contain 0b11 if the encoding is more than 64 bits.

\section{Out of Order Execution}
To support performance demands of modern processors, GeCo explicitly will have support for OOE(Out of Order Execution). With OOE instructions are not guaranteed any specific order, as long as it allows data to flow as intended by the program. We support 2 methods of OOE; VLIW, and instruction reordering. 
Both of these can allow the processor to take advantage of a pipeline for better execution.

\subsection{Instruction Reordering}
In instruction reordering, there is hardware inside the CPU that handles taking instructions, and moving them around to ensure their effects are correct, but that they run more efficiently.
This allows the process to avoid as many empty, or somehow unused slots in the pipeline as possible. 
The processor designers have the ultimate say in how exactly it will modify execution, as long as it doesn't change the result of the code.

\subsection{VLIW}
With VLIW the programmer defines sets of instructions which can execute in any order, these sets can be padded with nops if no other instructions can/should be executed in that set.
Unlike Instruction Reordering, in a VLIW implementation the CPU can not arbitrarily change around execution. 
The CPU must immediately start putting the set of instructions into the pipeline, at most the processor may attempt to reorder or ignore certain instructions in that set but may not move instructions or execute across multiple sets to avoid unexpected behaviour.

\section{Registers}
The \sname\ processors support 31 different general purpose registers, and 1 register doubling as a scratch, and no-access register. There are also 256 CSR (Control/Status Register) addresses.

\subsection{Definitions}
\begin{center}
\begin{tabularx}{\textwidth}{ |>{\raggedright\arraybackslash}X|>{\raggedright\arraybackslash}X|>{\raggedright\arraybackslash}X| }
    \hline
    Term & Read & Write \\
    \hline
    AA & Any & Any \\
    \hline
    AI & Any & Ignored \\
    \hline
    AL & Any & Legal \\
    \hline
\end{tabularx}
\end{center}

\subsection{Main Registers}
Main registers are intended to be used for general processing.
\begin{center}
\begin{tabularx}{\textwidth}{ |>{\raggedright\arraybackslash}X|>{\raggedright\arraybackslash}X|>{\raggedright\arraybackslash}X| }
    \hline
    Register ID & Name & Use \\
    \hline
    0b00000 & Zero/Scratch & Will read zero\footnotemark[1]\\
    \hline
    0b00001 & Stack base & Start of the stack in memory \\
    \hline
    0b00010 & Return address & Stores where to return to after a jump \\
    \hline
    0b00011 & gp0 & General purpose, any use \\
    \hline
    0b00100 & Frame pointer (IDR0) & Points to the start of the frame \\
    \hline
    0b00101 & Stack pointer (IDR0) & Current position in the stack \\
    \hline
    0b00110 & Global pointer (IDR0) & Points to the global storage region \\
    \hline
    0b00111 & Thread pointer (IDR0) & Points to the thread storage region \\
    \hline
    0b01000-0b01111 & hgp0-hgp7 & High access general purpose, used in cgc \\
    \hline
    0b10000-0b11111 & gp1-gp16 & General purpose, any use \\
    \hline
    \footnotetext[1]{It can be read/write if the trap bit in the processor flags is set, it's state must be consistent accross traps}
\end{tabularx}
\end{center}

\subsection{Control/Status Registers}
Contro/Status Registers are used to control aspects of the processor, or query the status of different parts of the processor.

\subsubsection{CSR Listing}
\begin{center}
\begin{tabularx}{\textwidth}{ |>{\raggedright\arraybackslash}X|>{\raggedright\arraybackslash}X|>{\raggedright\arraybackslash}X|>{\raggedright\arraybackslash}X|>{\raggedright\arraybackslash}X|>{\raggedright\arraybackslash}X| }
    \hline
    Address & Bits & Name & Access & Minimum Mode & Section \\
    \hline
    0x000000 & 64 & scause & AA & Supervisor & 9.2 \\
    \hline
    0x000001 & 64 & strap & AA & Supervisor & 9.2 \\
    \hline
    0x000002 & 64 & sret & AA & Supervisor & 9.2 \\
    \hline
    0x000003 & 64 & sscratch & AA & Supervisor & 9.2 \\
    \hline
    0x010000 & 64 & memmode & AA & Supervisor & 10 \\
    \hline
    0x010014 & 64 & memreg & AA & Supervisor & 10 \\
    \hline
\end{tabularx}
\end{center}

\section{Traps}
On a GeCo system traps serve to help handle internal errors, and external devices.

\subsection{Overview}
To provide proper access to external devices, and internal error handling, GeCo provides traps as they come in. These traps can occur due to several reasons, and can be mapped to different values.
\begin{center}
\begin{tabularx}{\textwidth}{ |>{\raggedright\arraybackslash}X|>{\raggedright\arraybackslash}X|>{\raggedright\arraybackslash}X| }
    \hline
    Source & Number & Reason \\
    \hline
    External (1) & 0x00 & MSI \\
    External (1) & 0x02 & Timer \\
    External (1) & 0x04 & External device \\
    \hline
    Internal (0) & 0x00 & Misaligned instruction \\
    Internal (0) & 0x02 & Misaligned read \\
    Internal (0) & 0x04 & Misaligned write \\
    Internal (0) & 0x06 & Invalid instruction address \\
    Internal (0) & 0x08 & Invalid read address \\
    Internal (0) & 0x0A & Invalid write address \\
    Internal (0) & 0x0C & Instruction page fault \\
    Internal (0) & 0x0E & Read page fault \\
    Internal (0) & 0x10 & Write page fault \\
    \hline
\end{tabularx}
\end{center}

\subsection{CSRs}
To allow for changing the device's state for traps there are several CSRs provided by GeCo. These CSRs can manage the trap handler address, the trap cause, the MSI source, and return address.

\subsubsection{strap}
Supervisor trap, or "strap", is a CSR used to control the virtual address of the trap handler, if virtual addressing is disabled then this address will be treated as physical.

\begin{center}
\begin{tabularx}{\textwidth}{ |>{\raggedright\arraybackslash}X|>{\raggedright\arraybackslash}X|>{\raggedright\arraybackslash}X| }
    \hline
    Bit(s) & Name & Use \\
    \hline
    0..62 & Strap Vaddr & Lower 63 bits of the trap handler's address \\
    \hline
    63 & Strap On & Enables trapping\footnotemark[1] \\
    \hline
\end{tabularx}
\end{center}
\footnotetext[1]{If a trap occurs and the strap.on bit is not set, then the core may repeatedly trap until powered down, halt execution but remain powered on, or it can power down that core, if every core in a chip is powered down then the entire system must power down. Regardless of if it shuts the core down, if other cores exist the CCD (specified in 7.3) must be notified}

\subsection{Core Communication Device}
The CCD (Core Communication Device) can be used to alert other cores of new events, this acts similarly to an MSI but may be implicitly triggered by the processor, or may have implicit reactions.
This device can be used to send IPIs, send remote fences, and change the state of other cores.

\subsubsection{Layout}
The CCD provides different regions, each region contains registers used for controlling the other cores, or the local core. It is split into 2 regions, the Core Control region (CCR), and the Core Local Buffer region (CLB).
The CCR must always be provided at physical address 0x2000, and take up 0x2000 bytes. The CLB must be provided at physical address 0x1000, and take up 0x1000 bytes.
\begin{center}
Core Control Region Layout
\begin{tabularx}{\textwidth}{ |>{\raggedright\arraybackslash}X|>{\raggedright\arraybackslash}X|>{\raggedright\arraybackslash}X|>{\raggedright\arraybackslash}X| }
    \hline
    Byte & Size & Name & Use \\
    \hline
    0x0 & 2 & Core 0 Control & Used to send a signal to the core\footnotemark[1] \\
    \hline
    ... & ... & ... & ... \\
    \hline
    0x1FF8 & 2 & Core 4095 Control & Used to send a signal to the core\footnotemark[1] \\
    \hline
\end{tabularx}
\end{center}
\footnotetext[1]{If a core is not associated with that register, nothing should happen}

\section{Virtual Addressing}

\end{document}